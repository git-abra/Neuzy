\documentclass[12pt,oneside]{article}

%%%%%%%%%%%%%%%%%%%%%%%%%%%%
%%   Zusaetzliche Pakete  %%
%%%%%%%%%%%%%%%%%%%%%%%%%%%%
\usepackage{setspace}
\usepackage{enumerate}  
\usepackage{fancyhdr}
\usepackage{a4wide}
\usepackage{graphicx}
\usepackage{palatino}
\usepackage{multirow}
\usepackage{booktabs}
\usepackage{titlesec}
\usepackage{enumitem}% http://ctan.org/pkg/enumitem

%folgende Zeile auskommentieren für englische Arbeiten
\usepackage[ngerman]{babel}
%folgende Zeile auskommentieren für deutsche Arbeiten
%\usepackage[ngerman, english]{babel}

\usepackage[T1]{fontenc}
\usepackage[utf8]{inputenc}
\usepackage[bookmarks]{hyperref}
\usepackage[justification=centering]{caption}
\usepackage[style=authoryear,natbib=true,backend=biber,maxbibnames=20]{biblatex}
\usepackage{csquotes}
\bibliography{literatur}

\setlength{\parindent}{0em} 
\setlist[itemize]{noitemsep, topsep=0pt}
\setlist[enumerate]{noitemsep, topsep=0pt}


%%%%%%%%%%%%%%%%%%%%%%%%%%%%%%
%% Definition der Kopfzeile %%
%%%%%%%%%%%%%%%%%%%%%%%%%%%%%%

\pagestyle{fancy}
\fancyhf{}
\cfoot{\thepage}
\setlength{\headheight}{16pt}

%%%%%%%%%%%%%%%%%%%%%%%%%%%%%%%%%%%%%%%%%%%%%%%%%%%%%
%%  Definition des Deckblattes und der Titelseite  %%
%%%%%%%%%%%%%%%%%%%%%%%%%%%%%%%%%%%%%%%%%%%%%%%%%%%%%

\newcommand{\JLUTitle}[9]{

  \thispagestyle{empty}
  \vspace*{\stretch{1}}
  {\parindent0cm
  \rule{\linewidth}{.7ex}}
  \begin{flushright}
    \vspace*{\stretch{1}}
    \sffamily\bfseries\Huge
    #1\\
    \vspace*{\stretch{1}}
    \sffamily\bfseries\large
    #2\\
    \vspace*{\stretch{1}}
    \sffamily\bfseries\small
    #3
  \end{flushright}
  \rule{\linewidth}{.7ex}

  \vspace*{\stretch{1}}
  \begin{center}
    \includegraphics[width=2in]{JLU_Logo} \\
    \vspace*{\stretch{1}}
    \Large Masterarbeit  \\

    \vspace*{\stretch{2}}
   \large Bioinformatik \& Systembiologie\\
    \large Justus-Liebig-Universität Gießen\\
    \vspace*{\stretch{1}}
    \large Betreuer:  #8 \\[1mm]
    
    \vspace*{\stretch{1}}
    \large Gießen, den #7
  \end{center}
}

\titlespacing*{\section}
{0pt}{3.5ex plus 1ex minus .2ex}{.2ex plus .2ex}
\titlespacing*{\subsection}
{0pt}{1.5ex plus 1ex minus .2ex}{.2ex plus .2ex}
\titlespacing*{\subsubsection}
{0pt}{1.5ex plus 1ex minus .2ex}{.2ex plus .2ex}

%%%%%%%%%%%%%%%%%%%%%%%%%%%%
%%  Beginn des Dokuments  %%
%%%%%%%%%%%%%%%%%%%%%%%%%%%%

\begin{document}
\onehalfspacing

  \JLUTitle
      {Validierung hippocampaler CA1 Pyramidenzellen mit HippoUnit }        % Titel der Arbeit
      {Adrian R}                        % Vor- und Nachname des Autors
      {MATRIKELNUMMER}
      
      {Fachbereich 08 - Biologie und Chemie}    % Name der Fakultaet
      {Gießen 2021}                             % Ort und Jahr der Erstellung
      {31.09.2021}                              % Tag der Abgabe
      {Prof. Dr. }                              % Name des Erstgutachters
      {Prof. Dr. }                              % Name des Zweitgutachters
      
  \clearpage

\lhead{}
\pagenumbering{Roman} 
    \setcounter{page}{1}

\tableofcontents
\clearpage

\addcontentsline{toc}{section}{\listfigurename}
\listoffigures

\addcontentsline{toc}{section}{\listtablename}
\listoftables
\clearpage

\setlength{\parskip}{0.5em} 


%%%%%%%%%%%%%%%%%%%%%%%%%%%%
%%  Kurzzusammenfassung   %%
%%%%%%%%%%%%%%%%%%%%%%%%%%%%
\lhead{Abstract}
\section*{Abstract}

Eine Kurzzusammenfassung der Vorgehensweise und der wesentlichen Ergebnisse.

Allgemeine Merkmale
\begin{itemize}
    \item Objektivität: Es soll sich jeder persönlichen Wertung enthalten.
    \item Kürze: Es soll so kurz wie möglich sein.
    \item Verständlichkeit: Es weist eine klare, nachvollziehbare Sprache und Struktur auf.
    \item Vollständigkeit: Alle wesentlichen Sachverhalte sollen enthalten sein.
    \item Genauigkeit: Es soll genau die Inhalte und die Meinung der Originalarbeit wiedergeben.
\end{itemize}{}


%%%%%%%%%%%%%%%%%%%%%%%%%%%%
%%  Einstellungen  %%
%%%%%%%%%%%%%%%%%%%%%%%%%%%%
\clearpage
\pagenumbering{arabic}  
    \setcounter{page}{1}
\lhead{\nouppercase{\leftmark}}

%%%%%%%%%%%%%%%%%%%%%%%%%%%%
%%  Hauptteil  %%
%%%%%%%%%%%%%%%%%%%%%%%%%%%%

\section{Abschnitt} \label{einleitung}
Ein Hauptabschnitt - idealerweise sollten keine Abschnitte leer sein.

\subsection{Unterabschnitt}
Ein Unterabschnitt - idealerweise sollten keine Abschnitte leer sein.

\subsubsection{Unterunterabschnitt}
Ein Unterunterabschnitt - idealerweise sollten keine Abschnitte leer sein.

\section{Einfache Formatvorlagen}

\textbf{Das ist fett gedruckter Text}.

\textit{Das ist kursiver Text}.


Auflistungen sind oft hilfreich für die Strukturierung:
\begin{itemize}
    \item Erster Eintrag
    \item Zweiter Eintrag
\end{itemize}

Nummerierte Aufzählungen sind oft hilfreich für Reihenfolgen:
\begin{enumerate}
    \item Erster Eintrag
    \item Zweiter Eintrag
\end{enumerate}


\section{Zitieren und Referenzieren}

Beiträge in Fachzeitschriften wie \citet{clemen1989combining} oder Konferenzartikel wie \citet{he2017mask} werden auf diese Weise im Text zitiert. In anderen Fällen möchte man aber in Klammern zitieren \citep{clemen1989combining}, auch mit mehreren Autoren \citep{clemen1989combining,baumol1958warehouse,he2017mask}.

Bei Monographien muss eine Seitenzahl mit angegeben werden \citep[S. 28]{chollet2018deep}.

So wird eine Webquelle zitiert: \citet{shiny1}. Es kann bei kurzen Informationen im Internet aber auch reichen die Adresse\footnote{\url{https://shiny.rstudio.com/tutorial/written-tutorial/lesson1/}} als Fußnote einzubetten.

So werden andere Teile der Arbeit referenziert: Kapitel \ref{einleitung}, Gleichung \ref{eq:1} zeigt...

So verweisen wir auf eine Fußnote \footnote{dies ist eine Fußnote}.

\section{Abbildungen}

Abbildungen erfordern das package \textit{graphicx}. 
Idealerweise verwendet man Vektorgrafiken oder hochaufgelöste Bitmaps. 
Eine gute Variante ist das Verwenden von PDFs.

\begin{figure}[h]
    \centering
    \includegraphics[width=0.3\textwidth]{JLU_Logo.png}
    \caption{Siegel der Universität}
    \label{fig:my_label}
\end{figure}


\section{Tabellen}

Die Tabular-Umgebung gibt die Anzahl Spalten an, deren Orientierung, Breite und evtl. Zwischenlinien. 


\begin{table}[ht]
    \centering
    \caption{Meine Tabelle}
        \begin{tabular}{ cccc } 
        \toprule
        col1 & col2 & col3 \\
        \midrule
        \multirow{3}{4em}{Multiple row} & cell2 & cell3 \\ 
        & cell5 & cell6 \\ 
        & cell8 & cell9 \\ 
        \bottomrule
    \end{tabular}
    \label{tab:countries}
\end{table}

\section{Formeln}

\begin{equation}
    \sum_{i=1}^N x_i
    \label{eq:1}
\end{equation}



%%%%%%%%%%%%%%%%%%%%%%%%%%%%
%% Literaturverzeichnis wird 
%% automatisch eingefügt
%%%%%%%%%%%%%%%%%%%%%%%%%%%%
\clearpage
\lhead{}
\printbibliography
\addcontentsline{toc}{section}{\bibname}


%%%%%%%%%%%%%%%%%%%%%%%%%%%%
%% Anhang (optional) 
%%%%%%%%%%%%%%%%%%%%%%%%%%%%
\clearpage
\appendix
\section{Anhang A}

%%%%%%%%%%%%%%%%%%%%%%%%%%%%
%% Eidesstattliche Erklärung
%% muss angepasst werden 
%% in Erklaerung.tex
%%%%%%%%%%%%%%%%%%%%%%%%%%%%
\newpage
\begin{otherlanguage}{ngerman}
\thispagestyle{empty}
\section*{Eidesstattliche Erklärung}
\thispagestyle{empty}
AAAHiermit versichere ich, die vorliegende Arbeit selbstständig verfasst und keine anderen als die angegebenen Quellen und Hilfsmittel benutzt sowie die Zitate deutlich kenntlich gemacht zu haben.
\newline
Ich erkläre weiterhin, dass die vorliegende Arbeit in gleicher oder ähnlicher Form noch nicht im Rahmen eines
anderen Prüfungsverfahrens eingereicht wurde.
\vspace{4\baselineskip}\\
Würzburg, den \today \hfill VORNAME NACHNAME 
\vspace{4\baselineskip}\\
\end{otherlanguage}


\end{document}
